\documentclass[12pt]{article}

\usepackage[russian]{babel}
\usepackage{graphicx}
\usepackage{wrapfig}
\usepackage{epigraph}


\begin{document}

{\underline{\bf Первое начало термодинамики. Теплоёмкость.}

{\underline{\bf Адиабатический и политропический процессы.}\\

Задачи: 1.40, 1.54, 1.87, 2.6}\\

{\bf ЗАДАНИЕ:1.100, 1.47, 1.75, 1.83}

\vspace{0.5cm}

Любая физическая теория - это модель, имеющая свои пределы применимости. Термодинамика (ТД) это не
теория, это эмпирическая наука о движении тепла. Теория - это статфизика.\\

В термодинамике под состоянием понимают набор параметров (свойств) системы, количественно характеризующих эту систему. В качестве набора параметров в основном используют четыре величины: давление (Р), температуру (Т), объем (V), энтропию (S).\\

Функция состояния - количественная характеристика системы, которая зависит только от равновесного состояния системы и не зависит от того, по какой траектории в пространстве параметров достигнуто это состояние.\\

В ТД одной из основных функций состояния тел является {\bf температура}.\\

{\bf Температура: \\

- определяется кинетической энергией {\underline{хаотического}} движения молекул $\frac{mv^2}{2}=i\frac{k_B T}{2}$,
где i-число степеней свободы и

$k_B=1.38\cdot10^{-23}$ Дж/К - постоянная Больцмана.\\

-непосредственно не измеряется, нужен термометр}\\

{\bf - Законы ТД применимы только к системам, находящимся в состоянии ТД равновесия, т.е. механического, теплового, химического и фазового равновесия.}\\

1Дж=1Н$\cdot$1м=0.1кгс$\cdot$м=1кг$\cdot$м$^2$/с$^2$=1Вт$\cdot$с=1Кл$\cdot$В=10$^7$эрг=6.2$\cdot$10$^{18}$эВ=0.24калории.\\

Для идеального газа внутренняя энергия $U=\frac{mv^2}{2}N=3\frac{kT}{2}N=\frac{3}{2}kT\cdot N=\frac{3}{2}\frac{m}{\mu}RT$, где

Газовая постоянная R=kN$_A$=8.3Дж/(моль$\cdot$К)

\vspace{0.5cm}

1)

{\bf 0 Начало:} Изолированная ТД система самопроизвольно переходит в состояние ТД равновесия.\\

{\bf 1 Начало:}$\delta Q=dU+\delta A$, где А-работа системы. Это обобщение закона сохранения энергии.\\

Если $dV\neq0$, то $\delta A=PdV$\\

Для идеального газа $\Delta U=\frac{m}{\mu}C_V \Delta T$ \\т.к. U зависит только от температуры U=U(T)\\

2) {\bf Теплоемкость  — количество теплоты, поглощаемой (выделяемой) телом в процессе нагревания (остывания) на 1 кельвин.}\\

$C=\frac{\delta Q}{dT}=\frac{dU}{dT}+P\frac{dV}{dT}$, где C может иметь любую величину и знак!\\

Теплоемкость зависит от процесса при котором происходит передача тепла.\\

- T=const $\Rightarrow dT=0 \Rightarrow C=\pm\infty$;\\

- V=const $\Rightarrow C=\frac{dU}{dT}=C_v$ \Rightarrow $dU=C_V dT$;$ т.к. $\frac{dV}{dT}=0$ при V=const.\\

- P=const $\Rightarrow C=C_p=C_v+P\frac{dV}{dT}=C_v+R$ для идеального газа.\\

\begin{figure}[h]
\begin{center}
\includegraphics[width=6cm]{cp_cv}
\end{center}
\end{figure}


3) {\bf Адиабата - ТД процесс без обмена теплотой:} $\delta Q=0 \Rightarrow \Delta U=-A$, где А-работа системы.\\

4) {\bf Политропа - ТД процесс с постоянной теплоемкостью С=const} Это, например, изотерма, адиабата, изохора;\\

Уравнение политропы для идеального газа: $PV^n=const$, где $n=\frac{C-C_p}{C-C_v}$ или $TV^{n-1}=const$;\\


{\bf Теплота} — это кинетическая часть внутренней энергии вещества, определяемая интенсивным хаотическим движением молекул и атомов, из которых это вещество состоит.

Теплота — это энергия (механическая или электромагнитная-излучение), переданная в ходе теплообмена.

Энергия может передаваться излучением от одного тела к другому и без их непосредственного контакта.

Количество теплоты не является функцией состояния.

Количество теплоты, полученное (отданное) системой в каком-либо процессе, зависит от способа, которым система была переведена из начального состояния в конечное.

========================================================================

\newpage

{\underline\bf Задача 1-40}

\vspace{0.5cm}

\begin{figure}[h]
%\begin{center}
\includegraphics[width=6cm]{1_40_0}
\end{figure}

\vspace{0.5cm}

1 начало: $\delta Q=dU+Pdv$

Политропа: Теплоемкость $C=const=\frac{\delta Q}{dT}=\frac{dU}{dT}+P\frac{dV}{dT}$\\

В общем случае внутренняя энергия U зависит от двух переменных U=U(T,V) или U=U(T,P) и $dU=(\frac{\partial U}{\partial T})_V dT+(\frac{\partial U}{\partial V})_T dV$;\\

Для идеального газа взаимодействие молекул отсутствует, поэтому U=U(T) только и $dU=(\frac{\partial U}{\partial T})_v dU=C_V dT;\\

Тогда $C=C_V+P\frac{dV}{dT}$;\\

Уравнение политропы: $PV^n=const$, где $n=\frac{C-C_P}{C-C_V}$ \Rightarrow\\ C(n-1)=nC_V-C_P$ и $P\frac{dV}{dT}=-\frac{PV}{T}=-R$;\\

$C=\nu(\frac{nC_V-C_P}{n-1})$; \\

Для моля идеального газа $U=i\frac{RT}{2}$, где i=3 для одноатомного газа и $C_V=\frac{3}{2}R $\\

Дифференциируя уравнение политропы по Т, получим: \\
$\frac{dP}{dT}V^n+nPV^(n-1)\frac{dV}{dT}=0$, откуда $\frac{dP}{P}+n\frac{dV}{V}=0 \Rightarrow$\\

$n=-\frac{ln\frac{P_1}{P_0}}{ln\frac{V_1}{V_0}}=\frac{ln8}{ln4}=\frac{3ln2}{2ln2}\frac{3}{2}\Rightarrow$\\

$n=\frac{\frac{3}{2}C_V-C_P}{\frac{3}{2}-1}=3C_V-2C_P=C_V-2R=-0.5R$, т.к. $C_P=C_V+R$ для идеального газа.\\

C=-0.5R для одного моля гелия.\\

$\nu=\frac{P_1V_1}{RT_1}=\frac{4l\cdot1atm}{0.082\frac{l\cdot atm}{mol\cdot K}300 K}=0.16$ моль.\\

$C=-0.16\cdot 0.5\cdot 2\frac{calor}{mol\cdot K}=-0.16\frac{calor}{K}$:

\newpage


{\underline\bf Задача 1-54}

\begin{figure}[h]
%\begin{center}
\includegraphics[width=6cm]{1_54_0}
\end{figure}

\begin{wrapfigure}[13]{l}{0.5\linewidth}
\vspace{ -4ex}
\includegraphics[width=4cm]{1_54_1}
%\caption()
\end{wrapfigure}

1) $C=const=\frac{\delta Q}{dT}=\frac{dU}{dT}+P\frac{dV}{dT}=C_V+P\frac{dV}{dT}$\\

$F=PS=kX \Rightarrow SdP=kdX=k\frac{dV}{S} \Rightarrow P=\frac{kV}{S^2}$ ;\\

2) $PV=RT \Rightarrow VdP+PdV=RdT$. Деля на PV получим\\

$\frac{dP}{P}+\frac{dV}{V}=\frac{RdT}{PV}=\frac{dT}{T}$\\

$P_0S^2=kV_0 \Rightarrow dP=\frac{kdV}{S^2} \Rightarrow \frac{dP}{P}=\frac{k}{S^2}\frac{dV}{\frac{kV}{S^2}}$\\

$\frac{dP}{P}=\frac{dV}{V}\Rightarrow 2\frac{dV}{V}=\frac{dT}{T} \Rightarrow \frac{dV}{dT}=\frac{V}{2T}$\\

$\frac{dV}{dT}=\frac{V}{2T}$\\

3) $C=C_V+P\frac{dV}{dT}=C_V+\frac{PV}{2T}=C_V+\frac{R}{2}$;\\

$C=C_V+\frac{R}{2}$;


\newpage


{\underline\bf Задача 1-87}


\begin{figure}[h]
%\begin{center}
\includegraphics[width=6cm]{1_87_0}
\end{figure}

1 начало: $\delta Q=dU+\delta \zetaA=dU+PdV=0; \Rightarrow dU=-PdV=-\frac{RTdV}{V}$;\\

1) Газ совершает работу, за счет уменьшения внутренней энергии, т.к. обмена теплом нет $\delta Q=0$;\\

2) В связи с теплообменом через перегородку AB, внутренняя энергия уменьшается как в водороде, так и в гелии, поэтому:\\

$dU=(Cv (H_2)+C_v (He))dT=(\frac{5}{2}+\frac{3}{2})RdT=4RdT$;\\

$4RdT=-\frac{RTdV}{V} \Rightarrow \frac{dT}{T}=-\frac{1}{4}\frac{dV}{V}\Rightarrow \ln\frac{T_1}{T_0}=-\frac{1}{4}\ln\frac{V_1}{V_0}=-\frac{1}{4}\ln2$;\\

$T_1=\frac{T_0}{\sqrt[4]{2}}$;



\newpage

{\underline\bf Задача 2-6}

\begin{figure}[h]
%\begin{center}
\includegraphics[width=13cm]{2_6_0}
\end{figure}

\vspace{1cm}

$V_s=\frac{1}{\sqrt{\rho\beta}},$ где $\beta=\frac{1}{\rho}(\frac{d\rho}{dP})_S$ - адиабатический коэффициент сжатия;\\

$V_s=\frac{1}{\sqrt{(\frac{d\rho}{dP})_S}}=\sqrt{(\frac{dP}{d\rho})_S}$\\

Адиабата: $PV^{\gamma}=const; \gamma=\frac{C_P}{C_V}; \rho=\frac{m}{V} \Rightarrow \frac{P}{\rho^{\gamma}}=const$\\

$P=A\rho^{\gamma}; \frac{\partial P}{\partial \rho}=A\gamma\rho^(\gamma-1)=\frac{\gamma P}{\rho}$;\\

$V_s=\sqrt{\frac{\gamma P}{\rho}}=\sqrt{\gamma\frac{m}{\mu}\frac{RT}{m}\frac{V}{V}}=\sqrt{\frac{\gamma RT}{\mu}}$;\\

$V_s^2=\frac{\gamma RT}{\mu}$;\\

$V^2_{He}-V^2_{sm}=(V_{He}-V_{sm})(V_{He}+V_{sm})\simeq 2V_{He}(V_{He}-V_{sm})$;\\

$\triangleV_s=(V_{He}-V_{sm})=\frac{V^2_{He}-V^2_{sm}}{2V_{He}}$\\

$\frac{\triangle V_s}{V_s}=\frac{1}{2}\left(1-\frac{V^2_{sm}}{V^2_{He}} \right)$;

$\frac{1}{\mu_{sm}}=\frac{1}{m_{sm}}\Sigma_i\frac{m_i}{\mu_i}=\Sigma_i\frac{\alpha_i}{\mu_i}$, где $\alpha_i=\frac{m_i}{m_{sm}}$\\

$\gamma_{sm}=\frac{C_{P sm}}{C_{V sm}}=\frac{\nu_{He}C_{P He}+\nu_{Ar}C_{P Ar}}{\nu_{He}C_{V He}+\nu_{Ar}C_{V Ar}}$\\

V^2_{sm}=RT\frac{\nu_{He}+\nu_{Ar}}{\nu_{He}\mu_{He}+\nu_{Ar}\mu_{Ar}}\frac{\nu_{He}C_{P He}+\nu_{Ar}C_{P Ar}}{\nu_{He}C_{V He}+\nu_{Ar}C_{V Ar}}\simeq 1.01RT;\\

$V^2_{s He}\simeq1.1 RT$

$\frac{\triangle V_s}{V_s}=\frac{1}{2}\left(1-\frac{V^2_{sm}}{V^2_{He}} \right)=\frac{1}{2}\left(1-\frac{1.01}{1.1}\right)=0.04$;\\

$\frac{\triangle V_s}{V_s}\simeq 0.04$ для $1\%$ Ar.

\end{document} 