\documentclass[12pt]{article}
\usepackage{mathtext}
\usepackage{hyperref}
\usepackage[warn]{mathtext}
\usepackage{mathtext}
\usepackage[T2A]{fontenc}
\usepackage[utf8]{inputenc}
\usepackage[russian]{babel}
\usepackage{graphicx}
\usepackage{wrapfig}


\begin{document}

{\underline{\bf Изменение энтропии в необратимых процессах. }

{\underline{\bf ТД потенциалы.}\\

Задачи:4.75, 4.43/44, 5.75, 5.38  \\

{\bf ЗАДАНИЕ:4.47, Т6, 5.32, 5.54}

\vspace{0.5cm}

{\bf 1) Изменение энтропии в необратимых процессах. }\\

В обратимых процессах $S_2-S_1=\int\limits_1^2\frac{\delta Q}{T}$ \\

В необратимых процессах $S_2-S_1>\int\limits_1^2\frac{\delta Q}{T}$ \\

Обратимым процессом называется переход системы из одного равновесного состояния в другое, который можно проводить в обратном направлении в той же последовательности промежуточных равновесных состояний. При этом система вместе с окружающими телами возвращаются к исходному состоянию.\\

Если система находится в состоянии равновесия во время процесса, она называется квазистатической.\\

Когда рабочее тело тепловой машины контактирует с тепловым резервуаром, температура которого неизменна во время всего процесса, то только изотермический квазистатический процесс считается обратимым, так как протекает с бесконечно малой разницей температур рабочего резервуара. Если имеется два резервуара, причем с разными температурами, тогда обратимым путем можно провести процессы на двух изотермических участках.\\

Так как адиабатический процесс проводится в обоих направлениях (сжатие и расширение), наличие кругового процесса с двумя изотермами и двумя адиабатами (цикл Карно)  это единственный обратимый круговой процесс, где рабочее тело контактируется с двумя тепловыми резервуарами. Остальные при наличии  тепловых резервуаров считаются необратимыми.\\

Превращение механической работы во внутреннюю энергию считается необратимым при наличии сил трения, диффузии  в газах и жидкостях. Все реальные процессы считаются необратимыми, даже если их значения  максимально приближены к обратимым. Обратимые процессы рассматриваются лишь как пример реальных процессов.\\

Большинство тепловых процессов протекают в одном направлении. Самопроизвольной передачи тепла от тела с низкой температурой к телу с высокой не наблюдается. Отсюда следует, что теплообмен  является необратимым процессом. \\

{\bf Причины необратимости: трение, теплопередача, диффузия, дросселирование и др.}\\

Все самопроизвольно протекающие процессы в изолированных термодинамических процессах характеризуются ростом энтропии.\\



{\bf 2) ТД потенциалы - функции, изменение которых не зависит от пути и определяется только начальным и конечным состоянием  системы. Это функции состояния.}\\

1. {\bf Внутренняя энергия: U=Q-A}.\\

2.{\bf Энтальпия:  H=U+PV}. 

При P=const $\Delta H= \Delta Q$\\

3. {\bf Свободная энергия Гельмгольца: F=U-TS} 

При T=const $\Delta F= \Delta A$, где A работа над внешними телами.\\

4. {\bf Потенциал Гиббса G=H-TS=F+PV=U+PV-TS}.\\

Соответствующие дифференциалы термодинамических потенциалов:

Внутренняя энергия: $dU=\delta Q-\delta A=TdS-PdV$\\

Энтальпия: $dH=dU+d(PV)=TdS-PdV+PdV+VdP=TdS+VdP$\\

Свободная энергия Гельмгольца: $dA=dU-d(TS)=TdS-PdV-TdS-SdT=-PdV-SdT$\\

Потенциал Гиббса: $dG=dH-d(TS)=TdS+VdP-TdS-SdT=VdP-SdT$\\

Эти выражения математически можно рассматривать как полные дифференциалы функций двух соответствующих независимых переменных U=U(S,V)}, H=H(S,P), F=F(T,V), G=G(T,P).\\

Задание любой из этих четырёх зависимостей - то есть конкретизация вида функций,  позволяет получить всю информацию о свойствах системы.\\

Известно, что  смешанные производные не зависят от порядка дифференцирования, то есть\\

${\frac {\partial U}{\partial V\partial S}}={\frac {\partial U}{\partial S\partial V}}}$.\\

Но {\left({\frac {\partial U}{\partial V}}\right)}_{S}=-P}$\\

 и {\left({\frac {\partial U}{\partial S}}\right)}_{V}=T}$, поэтому\\

${\left({\frac {\partial P}{\partial S}}\right)}_{V}=-{\left({\frac {\partial T}{\partial V}}\right)}_{S}}$.\\


Рассматривая выражения для других дифференциалов, получаем:\\


${\left({\frac {\partial T}{\partial P}}\right)}_{S}={\left({\frac {\partial V}{\partial S}}\right)}_{P}}$,\\

${\left({\frac {\partial S}{\partial V}}\right)}_{T}={\left({\frac {\partial P}{\partial T}}\right)}_{V}}$,\\

${\left({\frac {\partial S}{\partial P}}\right)}_{T}=-{\left({\frac {\partial V}{\partial T}}\right)}_{P}}$ \\

Эти соотношения называются соотношениями Максвелла.



========================================================================

\newpage

{\underline\bf Задача 4.75}

\vspace{0.5cm}

\begin{figure}[h]
%\begin{center}
\includegraphics[width=12cm]{4_75_0}
\end{figure}

\vspace{0.5cm}

Процесс неравновесный - система изолирована и самопроизвольно приходит в состояние теплового равновесия.\\

$\triangle S=\triangle S_1+\triangle S_2$;\\

$dS=C_P\frac{dT}{T}-R\frac{dP}{P}$\\

$\triangle S_1=C_P\ln\frac{T_2}{T_1}-R\ln2$, т.к. $P_2=2P_1$;\\

$\triangle S_2=C_P\ln\frac{T_3}{T_2}+R\ln2$, т.к. $P_2=2P_3$;\\

$\triangle S=\triangle S_1+\triangle S_2=C_P\ln\left(\frac{T_2}{T_1}\frac{T_3}{T_2}\right)=C_P\ln\left(\frac{T_3}{T_1}\right) $\\

$A_1=\triangle U=C_V(T_2-T_1)=2P_1(V_1-V_2)=2P_1V_1-P_2_V_2=2RT_1-RT_2$, обе части делим на $T_1 \Rightarrow$\\

$C_V\left(\frac{T_2}{N_1}-1 \right)=2R-R\frac{T_2}{T_1}\Rightarrow \frac{T_2}{T_1}=\frac{C_P+R}{C_P}$\\ 

$A_2=\triangle U=C_V(T_2-T_3)=P(V_3-V_2)=RT_3-(R/2)T_2 \Rightarrow \frac{T_3}{T_2}=\frac{C_P-R/2}{C_P}$\\

$\triangle S=C_P\ln\frac{(C_P+R)(C_P-R/2)}{C_P^2}=\frac{5}{2}R\ln\frac{28}{25}\simeq2.5$ Дж/К;

\newpage


{\underline\bf Задача 4.43/44}


\begin{figure}[h]
%\begin{center}
\includegraphics[width=12cm]{4_43_0}
\end{figure}

\vspace{1cm}


1) Смешивание газов процесс необратимый, поэтому $TdS\geq \delta Q=dU+\delta A$;\\

$T=const\Rightarrow dU=0\Rightarrow \delta A\leq TdS \Rightarrow$\\

$A\leq T(S_2-S_1)=RT\left(\nu_1\ln\frac{V_1+V_2}{V_1}+\nu_2\ln\frac{V_1+V_2}{V_2}\right)\simeq1.8$ кДж;\\

2) Адиабатическое смешивание: \\

$\delta Q=0 \Rightarrow A=\triangle U=\nu_1C_{V_1}(T_1-T)+\nu_2C_{V_2}(T_1-T)$\\

$\triangle S=\nu C_V\ln\frac{T}{T_1}+\nu R\ln\frac{V}{V_1}$;\\

$TdS=\delta Q=0 \Rightarrow$\\

$\nu_1C_{V_1}\ln\frac{T}{T_1}+\nu_1 R\ln\frac{V_1+V_2}{V_1}+\nu_2C_{V_2}\ln\frac{T}{T_2}+\nu_2 R\ln\frac{V_1+V_2}{V_2}=0$ Откуда\\

$T=T_1\left(\frac{V_1V_2}{(V_1+V_2)^2} \right)^{1/5}=0.75 T_1=225$ К;\\

$A_{max}=(\nu_1+\nu_2)C_V(T_1-T)=1.55$ кДж.

 
\newpage

{\underline\bf Задача 5.75 }

\begin{figure}[h]
%\begin{center}
\includegraphics[width=12cm]{5_75_0}
\end{figure}

\vspace{0.5cm}

$dФ=-SdT+VdP$;\\

$V=\left(\frac{\partial Ф}{\partial T} \right)_T=\frac{RT}{P} \Rightarrow PV=RT$ - 1 моль идеального газа;\\

$S=-\left(\frac{\partial Ф}{\partial T} \right)_P=a(1-\ln T)-\frac{aT}{T}+R\ln P-S_0=R\ln P-a\ln T=R\ln\frac{RT}{V}-a\ln T$;\\

$U=-T^2\left(\frac{\partial}{\partial T}\frac{F}{T} \right)$\\

$Ф=F+PV \Rightarrow F=Ф-PV; \frac{F}{T}=\frac{Ф}{T}-\frac{PV}{T}=a(1-\lnT)+R\ln P-S_0+\frac{U_0}{T}-\frac{PV}{T}$\\

$U=aT-VT\left( \frac{\partial P}{\partial T} \right)_V+U_0,  \frac{\partial P}{\partial T} \right)_V=\frac{R}{V} $;\\

$U=(a-R)T+U_0$\\

$dU=(a-R)dT=C_V dT, (a-R)=C_V\Rightarrow a=C_P$;\\

$H=U+PV=U+RT=aT+U_0$\\

$U=C_V T +U_0;  H+C_P+U_0; a=C_P$



\newpage


{\underline\bf Задача 5.38}

\begin{figure}[h]
%\begin{center}
\includegraphics[width=12cm]{5_38_0}
\end{figure}

\vspace{0.5cm}

$\delta Q=TdS-PdV=TdS$ т.к. dV=0\\

$dF=-SdT-PdV$\\

$F=F_0-\frac{\alpha}{T}B^2, T=const$;\Rightarrow\\

$S=-\left( \frac{\partial F}{\partial T} \right)_V=\frac{\alpha B^2}{T^2}$,  B меняется от $B_0$ до 0.

$\triangle S=\frac{\alpha B_0^2}{T^2}$\\

$\triangle Q=T\triangle S=\frac{\alpha B_0^2}{T}$








\end{document} 