\documentclass[12pt]{article}
\usepackage{mathtext}
\usepackage{hyperref}
\usepackage[warn]{mathtext}
\usepackage{mathtext}
\usepackage[T2A]{fontenc}
\usepackage[utf8]{inputenc}
\usepackage[russian]{babel}
\usepackage{graphicx}
\usepackage{wrapfig}


\begin{document}

{\underline{\bf Второе начало термодинамики. Тепловые машины.}

{\underline{\bf Изменение энтропии в тепловых процессах.}\\

Задачи:3.25, 3.43, Т1, 4.80 \\

{\bf ЗАДАНИЕ:3.52, 3.47, 4.15, 4.78}

\vspace{0.5cm}

{\bf 1) 2 начало.}\\

1. Для любой квазиравновесной ТД системы существует однозначная функция состояния - {\bf энтропия}\\

$dS=\frac{\delta Q}{T}$-приведенная теплота.

-В состоянии равновесия $S=S_{max}$;

-Энтропия изолированной системы может только увеличиваться $S\geq0$;

- В обратимых процессах $\Delta S=0$;\\

2. {\bf Постулат Клаузиуса:} Невозможен процесс, единственным результатом которого является передача тепла от холодного тела горячему.\\

3. {\bf Постулат Томпсона(Кельвина):} Невозможен процесс, единственным результатом которого является производство работы за счет охлаждения теплового резервуара.\\

4. {\bf Энтропия идеального газа.} Для одного моля:\\

$S(V,T)=S_0+C_V\ln\frac{T}{T_0}+R\ln\frac{V}{V_0}$

$S(T,P)=S^\prime_0+C_P\ln\frac{T}{T_0}-R\ln\frac{P}{P_0}$

$S(V,T)=S^{\prime\prime}_0+C_V\ln\frac{P}{P_0}+C_P\ln\frac{V}{V_0}$\\

2){\bf 3 начало. Теорема Нернста.}\\

Всякий термодинамический процесс, протекающий при фиксированной температуре T, сколь угодно близкой к нулю, {$T\to 0$}, не должен сопровождаться изменением энтропии S, то есть энтропия любой системы при абсолютном нуле температуры, T=0, является универсальной постоянной $S_{0}$, не зависящей ни от каких переменных параметров (давления, объема и т. п.). Энтропия принимает абсолютные значения.

Значение {\bf $S_0=0$} (Планк).\\

{\bf 3) Цикл Карно.}\\

\begin{figure}[h]
\begin{center}
\includegraphics[width=11cm]{karno}
\end{center}
\end{figure}



{\bf Процесс, называемый циклом Карно, превращает тепло в работу наиболее эффективным образом, т.е. с максимальным КПД $\eta$.}\\

Процесс представляет собой замкнутый цикл, состоящий из двух изотерм и двух адиабат, представленный на рисунке
в координатах

PV и TS, где S - энтропия. Из рисунка TS следует, что для любого цикла при тех же температурах нагревателя и холодильника $T_1$ и $T_2$ его площадь будет меньше приведенной, т.к. цикл будет вписан в прямоугольник и его КПД меньше КПД цикла Карно $\eta<\eta_K$.\\

По определению КПД $\eta=\frac{Q_1-Q_2}{Q_1}=1-\frac{Q_2}{Q_1}$;\\

Энтропия - функция состояния, поэтому ее изменение за цикл равно нулю. Тогда\\

$\oint dS=\int_1^2\frac{\delta Q}{T_1}-\int_3^4\frac{\delta Q}{T_2}=\frac{Q_1}{T_1}-\frac{Q_2}{T_2}=0 \Rightarrow \frac{Q_2}{Q_1}=\frac{T_2}{T_1} \Rightarrow $\\

$\eta=1-\frac{T_2}{T_1}$, где холодильником ($T_2$) может служить атмосфера.\\

Для паровой турбины $T_1=800$ К и $T_2=300$ К и значение максимального КПД для цикла Карно\\

$\eta_{max}=\frac{T_1-T_2}{T_1}=0.62=62\%$\\

Из за потерь реальное значение КПД $40\%$. Максимальное значение КПД - около $44\%$ - имеют двигатели внутреннего сгорания.\\


{\bf 4) Обратный цикл Карно - холодильник и тепловой насос.}


\begin{wrapfigure}[11]{l}{0.5\linewidth}
\includegraphics[width=6cm] {obr_cicl}
\caption{Обратный цикл Карно в P,V- и T, S-диаграммах.}
\end{wrapfigure}

Процесс, в котором теплота забирается у менее нагретого тела и отдается более нагретому телу в результате совершения работы над системой внешними телами, называется обратным. По обратному циклу работают холодильные машины и тепловые насосы.

Пусть цикл Карно идет в обратном направлении. Рабочее тело с начальными параметрами точки а расширяется адиабатно, совершая работу расширения за счет внутренней энергии, и охлаждается от температуры $Т_1$ до температуры $T_2$. Дальнейшее расширение происходит по изотерме, и рабочее тело отбирает от нижнего источника с температурой $T_2$ теплоту $Q_2$. Далее газ подвергается сжатию сначала по адиабате, и его температура от $Т_2$ повышается до $T_1$, а затем — по изотерме ($T_1$=const). При этом рабочее тело отдает верхнему источнику с температурой $T_1$ количество теплоты $Q_1$.

Описанный цикл теплового двигателя является полностью обратным циклом Карно. То есть все процессы, из которых он состоит, могут быть обращены вспять, и в этом случае цикл становится {\bf холодильным циклом Карно}.

Тепло поглощается из низкотемпературного резервуара и отдается в высокотемпературный резервуар.
{\bf Для выполнения такого процесса требуется выполнение  работы!}.

Поскольку в обратном цикле сжатие рабочего тела происходит при более высокой температуре, чем расширение, работа сжатия, совершаемая внешними силами, больше работы расширения на величину площади abcd, ограниченной контуром цикла. Эта работа превращается в теплоту и вместе с теплотой $Q_2$ передается верхнему источнику. Таким образом, затратив на осуществление обратного цикла работу $A_c$, можно перенести теплоту от источника с низкой температурой к источнику с более высокой температурой. При этом нижний источник отдаст количество теплоты $Q_2$, а верхний получит количество теплоты $Q_1$.  Работа $A_c=Q_l- Q_2$.

Обратный цикл Карно является идеальным циклом холодильных установок и так называемых тепловых насосов.


{\bf ХОЛОДИЛЬНИК}\\

{\bf 1. Какая работа совершается в холодильнике?}

\begin{wrapfigure}[13]{l}{0.5\linewidth}
\includegraphics[width=6cm] {holod_sh}
\caption{Схема работы холодильника.}
\end{wrapfigure}

Принцип работы заключается в том, что  влага при испарении поглощает тепло. А при конденсации, наоборот, тепло выделяется. В холодильных машинах  по замкнутому кругу двигается специальная жидкость (хладагент). Хладагент испаряется в испарителе и конденсируется в конденсаторе. При этом испаритель охлаждается, а конденсатор греется.

Чтобы хладагент испарялся и конденсировался в нужных местах, в холодильном контуре должны присутствовать два элемента – компрессор и дросселирующее устройство.

Компрессор сжимает газообразный хладагент в конденсаторе, где он под действием высокого давления переходит в жидкую форму, выделяя тепло. При этом компрессор выполняет работу $A_c$. Дросселирующее устройство  затрудняет движение хладагента и поддерживает высокое давление в конденсаторе. После дросселя давление в контуре намного ниже, и попавший туда хладагент начинает испаряться внутри испарителя, поглощая тепло. Далее он, уже в газообразном виде, снова попадает в компрессор, и цикл повторяется.

 При одинаковых условиях разные жидкости имеют разные температуры кипения, так, например, при нормальном атмосферном давлении вода закипает при температуре +100°С, этиловый спирт +78°С, фреон R-22 минус 40,8°С, фреон R-502 минус 45,6°С, фреон R-407 минус 43,56°С, жидкий азот минус 174°С.

 {\bf 2. КПД холодильника.}\\

Эффективность холодильной установки оценивается холодильным коэффициентом, определяемым как отношение количества теплоты, отнятой за цикл от холодильной камеры, к затраченной в цикле работе:\\

$\varepsilon=\frac{Q_2}{A}=\frac{Q_2}{Q_1-Q_2};\\

Холодильную установку можно использовать в качестве теплового насоса. Если $Q_2$ — количество теплоты, взятое от наружной атмосферы,  а А — расход электроэнергии, то количество теплоты приходящей в помещение $Q_1=Q_2+A$.

Характеристикой насоса является отопительный коэффициент\\

$\varepsilon_n=\frac{|Q_1|}{A}=\frac{T_1}{T_1-T_2}$

На фото изображен пример извлечения тепла $Q_2$ из  резервуара (грунт).

\begin{figure}[h]
\begin{center}
\includegraphics[width=6cm]{grunt}
\end{center}
\end{figure}

В трубах часто используют пропиленгликоль, который забирает тепло земли, передает его хладагенту, и остыв, снова отправляется в грунтовый коллектор.

Проблема: Чтобы обогреть дом 100 кв.м. потребуется около 5 соток на участке для коллектора, и над коллектором нельзя будет возводить капитальных строений и сажать деревья с мощной корневой системой.

Тепловой насос  выдает в 3-7 раз больше тепловой энергии, чем тратит электроэнергии.




========================================================================

\newpage

{\underline\bf Задача 3-25}

\vspace{0.5cm}

\begin{figure}[h]
%\begin{center}
\includegraphics[width=6cm]{3_25_0}
\end{figure}

\vspace{0.5cm}

Ответ: $A_{max}=\triangle Q_1-\triangle Q_2$, где $Q_1$ - теплота нагревателя, а $Q_2$ холодильника.\\

1) В процессе работы тепловой машины $T_1$ меняется и уменьшается до $T_k$, поэтому $\delta Q=Cm_1dT$\\

Интегрируя получим $\triangle Q_1=\int_{T_1}^{T_k}Cm_1dT=Cm_1(T_1-T_k)$;\\

$T_2=const$ и $\triangle Q_2=qm_2$. Откуда\\

$A_{max}=Cm_1(T_1-T_k)-qm_2$;\\ Остается найти $T_k$;\\

2) Энтропия в цикле Карно сохраняется, откуда $\frac{\triangle Q_1}{T_1}=\frac{\triangle Q_2}{T_2}\Rightarrow$;\\

$\frac{\triangle Q_2}{T_2}=\int_{T_1}^{T_k}\frac{\delta Q}{T}dT=-Cm_1\intt_{T_1}^{T_k}\frac{dT}{T}=-Cm_1\ln\frac{T_k}{T_1}$;\\

$\frac{qm_2}{T_2}=-Cm_1\ln\frac{T_k}{T_1}$ и \\

$T_k=T_1e^{-\frac{qm_2}{cm_1T_2}}=278$ К;\\

$A_{max}=cm_1T_1(1-e^{-\frac{qm_2}{cm_1T_2}})-qm_2=62$ кДж;\\

где $С\simeq4217$  Дж/(кг К) - удельная теплоемкость воды.




\newpage


{\underline\bf Задача 3-43}

\begin{figure}[h]
%\begin{center}
\includegraphics[width=6cm]{3_43_0}
\end{figure}

\vspace{0.5cm}

Процесс стационарный, поэтому рассматривать надо теплоту в единицу времени, т.е. мощность.\\

Введем обозначения:\\

$T_0$ - искомая температура в помещении при работе теплового насоса (ТН);\\

Мощность потока тепла из помещения на улицу пропорциональна разности температур с неизвестным коэффициентом k. Тогда \\

1. $N_0=k(T_0-T_2)$ - мощность ТН, поддерживающего разность температур $(t_x-t_2)$;\\

2. $N_1=k(T_1-T_2)$ - мощность горелки, поддерживающей разность температур $(t_1-t_2)$;\\

3. $N_2$ - мощность ТН, отбираемая с улицы. Тогда можно записать:\\

$N_0=N_2+\eta N_1$, где $\eta N_1$ - мощность горелки, передаваемя ТН;\\

Из цикла Карно $\frac{N_2}{T_2}=\frac{N_0}{T_0} \Rightarrow N_2=\frac{T_2}{T_0}N_0=\frac{T_2}{T_0}(N_2+\eta N_1)\Rightarrow$\\

$N_2(1-\frac{T_2}{T_0})=\eta N_1\frac{T_2}{T_0}=\eta k\frac{T_2}{T_0}(T_1-T_2)=\eta \frac{T_2}{T_0}\frac{N_0}{T_0-T_2}(T_1-T_2)$;\\

$N_2(T_0-T_2)^2=N_0\eta T_2(T_1-T_2); \frac{N_0}{N_2}=\frac{T_0}{T_2}\Rightarrow$\\

$N_2(T_0-T_2)^2=\eta T_0(T_1-T_2);$ Обозначим $T_0-T_2=X \Rightarrow$\\

$X^2-X\eta (T_1-T_2)-\eta T_2(T_1-T_2)=0$;\\

X=49 $\Rightarrow T_0=T_2+49 \Rightarrow$

$T_0=299$ К =$26^oC$









\newpage


{\underline\bf Задача Т-1}


\begin{figure}[h]
%\begin{center}
\includegraphics[width=11cm]{T_1_0}
\end{figure}

\vspace{1cm}

Эффективность холодильника оценивается холодильным коэффициентом $\varepsilon=\frac{Q_2}{A}\geq 1$, где $Q_2$ - тепло отведенное от холодильной камеры;\\

 $\varepsilon=\frac{Q_2}{A}=\frac{Q_2}{Q_1-Q_2}=\frac{T_2}{T_1-T_2}=2\Rightarrow A=\frac{Q_2}{2}$;\\

 $Q_2=C_V \triangle T=\frac{5}{2}R\triangle T=2.5\cdot 8.3\cdot100\simeq 2$ кДж;\\

 $А\simeq 1$ кДж;\\

 $\triangle Q_1=\triangle Q_2+A=\frac{3}{2}\triangle Q_2\Rightarrow \triangle T_1=\frac{3}{2}\triangle T_2=150$ К;\\

 $T^\prime_2=300+150=450$ К;

\newpage

{\underline\bf Задача 4-80}

\begin{figure}[h]
%\begin{center}
\includegraphics[width=11cm]{4_80_0}
\end{figure}

\vspace{0.5cm}

Механическое равновесие атмосферы или ее стационарность означает, что в тонком слое dh разность давлений P и P+dP на верхнем и нижнем уровнях \\

$dP=-\rho gdh$; Знак минус т.к. dP и dh направлены противоположно.\\

Атмосфера адиабатическая, значит на разных уровнях выполняется уравнение\\

$PV^{\gamma}=const$, обозначим эту константу буквой $A=PV^{\gamma}=P_0V_0^{\gamma}$;\\

По условию $C)_2$ газ идеальный, поэтому на любом уровне его состояние описывается уравнением \\

$PV=\frac{M}{\mu}RT$;\\

Используя уравнение адиабаты уравнение состояния газа можно записать в следующем виде:\\

$P_0(\left(\frac{A}{P_0})^{\frac{1}{\gamma}\right)=\frac{M}{\mu}RT_0,$ откуда $A=\left(\frac{MRT_0}{\mu}  \right)^{\gamma}P_0^{1-\gamma}$\\

Уравнение стационарности атмосферы можно переписать в следующем виде:\\

$dP=-\frac{M}{V}gdh=-\left(\frac{A}{P}\right)^{\frac{1}{\gamma}}=-\frac{M}{A^{\frac{1}{\gamma}}}P^{\frac{1}{\gamma}}gdh=$\\

$=-\frac{M}{\frac{MRT_0}{\mu}}P_0^{\frac{1-\gamma}{\gamma}}P^{\frac{1}{\gamma}}gdh=-BP^{\frac{1}{\gamma}}gdh$;\\

$\int\frac{dP}{P^{\frac{1}{\gamma}}}=-Bg\int dh$; Откуда $\frac{\gamma}{\gamma-1}P^{\frac{\gamma-1}{\gamma}}=-Bh +C$;\\

При $h=h_0, C=\frac{\gamma}{\gamma-1}P_0^{\frac{\gamma-1}{\gamma}}\Rightarrow$\\

$P^{\frac{\gamma-1}{\gamma}}=P_0^{\frac{\gamma-1}{\gamma}}-\frac{\gamma-1}{\gamma}\frac{B}{P_0^{\frac{\gamma-1}{\gamma}}}h$;\\

$\frac{B}{P_0^{\frac{R}{C_P}}}=\frac{\mu g}{RT_0,}$ Тогда \\

$P=P_0\left(1-\frac{R}{C_P}\frac{\mu gH}{RT_0} \right)^{\frac{C_P}{R}}$, т.к. $\frac{\gamma-1}{\gamma}=\frac{R}{C_P}$;\\

Из этого выражения получаем искомое значение $T_0$:\\

$T_0=\frac{R}{C_P}\frac{\mu gH}{R}\frac{1}{1-\left( \frac{P}{P_0} \right)^{\frac{R}{C_P}}}$\\

$P=\frac{\rho}{\mu}RT \Rightarrow \frac{P}{P_0}=\frac{\rho}{\rho_0}\frac{T}{T_0}=\frac{\rho}{\rho_0}\left(\frac{V_0}{V}\right)^{\gamma-1}=
\left(\frac{\rho}{\rho_0}\right)^{\gamma}\Rightarrow$\\

$T_0=\frac{\mu gH}{ C_P \left[1-\left( \frac{\rho}{\rho_0} \right)^{\frac{R}{C_V}} \right]}$


\end{document} 