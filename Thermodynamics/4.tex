\documentclass[12pt]{article}
\usepackage{mathtext}
\usepackage{hyperref}
\usepackage[warn]{mathtext}
\usepackage{mathtext}
\usepackage[T2A]{fontenc}
\usepackage[utf8]{inputenc}
\usepackage[russian]{babel}
\usepackage{graphicx}
\usepackage{wrapfig}


\begin{document}

{\underline{\bf Применение термодинамических потенциалов. }

{\underline{\bf Преобразования термодинамических функций.}\\

Задачи: 5.16, 5.28, 12.8, 5.42  \\

{\bf ЗАДАНИЕ:5.63, 5.40, 12.9, 12.38}

\vspace{0.5cm}

Идея метода состоит в том, что  термодинамические потенциалы (ТД), можно выразить через параметры системы, такие как P,V,T,S.  Пользуясь этим можно составить уравнения, необходимые для анализа того или иного явления.\\

{\bf ПРИМЕР:}\\

Для закрытой системы основное уравнение ТД имеет вид  : TdS=dU + PdV;\\

Оно связывает пять величин: T, S, U, P, V. Само же состояние простой системы определяется двумя параметрами. Поэтому, выбирая из пяти названных величин две в качестве независимых переменных, мы получаем, что основное уравнение содержит еще три неизвестные функции. 

Определение этих трех неизвестных величин упрощается с введением термодинамических потенциалов.

Выразим из основного уравнения  : dU=TdS-PdV. 

Приращение внутренней энергии полностью определяется приращением энтропии и приращением объема. Поэтому если  в качестве независимых переменных выберать S и V, то для определения других трех переменных нужно знать лишь одно уравнение U=U(S,V) для внутренней энергии как функции S и V.

Зная зависимость U(S,V) и основное уравнение, простым дифференциированием определяются P и V. \\

Вторые производные позволяют определить теплоемкости $C_P$, $C_V$ и модуль упругости. Смешанные производные позволяют установить соотношение между изменением температуры и адиабатическим расширением:\\

$\left( \frac{\partial T}{\partial V}\right)_S=-\left( \frac{\partial P}{\partial S}\right)_V$\\

Таким образом, внутренняя энергия  как функция переменных S и V, является характеристической функцией. Ее первые производные определяют термические свойства системы, вторые – калорические свойства системы, смешанные - соотношения между другими свойствами системы. Установление таких связей и составляет содержание метода ТД потенциалов. А U(S,V) является одним из множества ТД потенциалов.\\

Выражение для ТД потенциалов, его явный вид, можно найти только для 2-х систем, одной из которых является идеальный газ, другой равновесное излучение, т.к. для них известны и уравнения состояния и внутренняя энергия как функция параметров. \\

Для всех других систем ТД потенциалы находятся либо из опыта, либо методами статистической физики, и потом с помощью полученных ТД соотношений определяют уравнения состояния и другие свойства. \\

Для газов ТД функции чаще всего вычисляются методами статистической физики, для жидкостей и твердых тел они обычно находятся экспериментально с помощью калорических определений теплоемкости.\\

{\bf 1) ТД потенциалы. }\\

1. {\bf Внутренняя энергия: U=Q-A}.\\

2.{\bf Энтальпия:  H=U+PV}. При P=const $\Delta H= \Delta Q$\\

3. {\bf Свободная энергия Гельмгольца: F=U-TS} При T=const $\Delta F= \Delta A$, где A работа над внешними телами.\\

4. {\bf Потенциал Гиббса G=H-TS=F+PV=U+PV-TS}.\\

{\bf 2) Дифференциалы ТД потенциалов:}\\

Внутренняя энергия: $dU=\delta Q-\delta A=TdS-PdV$\\

Энтальпия: $dH=dU+d(PV)=TdS-PdV+PdV+VdP=TdS+VdP$\\

Свободная энергия Гельмгольца: $dA=dU-d(TS)=TdS-PdV-TdS-SdT=-PdV-SdT$\\

Потенциал Гиббса: $dG=dH-d(TS)=TdS+VdP-TdS-SdT=VdP-SdT$\\



{\bf 3) Соотношения Максвелла для ТД функций.}\\



1. ${\left({\frac {\partial P}{\partial S}}\right)}_{V}=-{\left({\frac {\partial T}{\partial V}}\right)}_{S}}$.\\


2. ${\left({\frac {\partial T}{\partial P}}\right)}_{S}={\left({\frac {\partial V}{\partial S}}\right)}_{P}}$,\\

3. ${\left({\frac {\partial S}{\partial V}}\right)}_{T}={\left({\frac {\partial P}{\partial T}}\right)}_{V}}$,\\

4. ${\left({\frac {\partial S}{\partial P}}\right)}_{T}=-{\left({\frac {\partial V}{\partial T}}\right)}_{P}}$ \\





========================================================================

\newpage

{\underline\bf Задача 5.16}

\vspace{0.5cm}

\begin{figure}[h]
%\begin{center}
\includegraphics[width=12cm]{5_16_0}
\end{figure}

\vspace{0.5cm}

$\alpha=\frac{1}{V}\left( \frac{\partial V}{\partial T} \right)_P$ - коэффициент объемного расширения;\\

Процесс адиабатический $S=const\Rightarrow$ Надо найти изменение температуры с давлением при постоянном S.\\

 $\left( \frac{\partial T}{\partial P} \right)_S-?$\\

1) $dS=\left( \frac{\partial S}{\partial T}\right)_P dT+\left( \frac{\partial S}{\partial P} \right)_T dP=0\Rightarrow$\\

 $\left( \frac{\partial T}{\partial P} \right)_S=\frac{\left( \frac{\partial S}{\partial P} \right)_T}{\left( \frac{\partial S}{\partial T}\right)_P}$;\\


$\left( \frac{\partial S}{\partial T}\right)_P=\frac{1}{T}\left( \frac{T\partial S}{\partial T}\right)_P=\frac{1}{T}\left( \frac{\delta Q}{\partial T}\right)_P=\frac{C_P}{T}$\\

$\left( \frac{\partial S}{\partial P}\right)_T=-\left( \frac{\partial V}{\partial T}\right)_P=-\alpha V$ - Соотношение Максвелла \\

2) $\left( \frac{\partial T}{\partial P} \right)_S=\left( \frac{\partial V}{\partial T}\right)_P\frac{T}{C_P}\Rightarrow$\\

$\triangle T=\int\limits_{P_1}^{P_2}\left( \frac{\partial V}{\partial T}\right)_P\frac{T}{C_P} dP=-\int\limits_{P_1}^{P_2}\frac{\alpha VT}{C_P} dP=-\int\limits_{P_1}^{P_2}\frac{\alphaVT}{\rho C_P} dP$, т.к. для уд.теплоемкости $V=1/\rho$;\\

$\triangle T=-\frac{\alpha T}{\rho C_P}\Delta P=-0.26^o C$;





\newpage


{\underline\bf Задача 5.28}


\begin{figure}[h]
%\begin{center}
\includegraphics[width=12cm]{5_28_0}
\end{figure}

\vspace{1cm}

1) Между представлеными коэффициентами существует простое соотношение $\alpha=-\kappa\beta_T$, которое можно получить, используя известное соотношение между производными:\\

$\left( \frac{\partial V}{\partial T} \right)_P\left( \frac{\partial T}{\partial P} \right)_V\left( \frac{\partial P}{\partial V} \right)_T=-1;\Rightarrow$\\

$\left( \frac{\partial V}{\partial T} \right)_P= -\frac{1}{\left( \frac{\partial T}{\partial P} \right)_V\left( \frac{\partial P}{\partial V} \right)_T}=-\left( \frac{\partial P}{\partial T} \right)_V\left( \frac{\partial V}{\partial P} \right)_T\Rightarrow$\\

$V\alpha=-\kappa V\beta_T\Rightarrow \alpha=-\kappa\beta_T$;\\

2) T=const - изотерма;\\

Надо найти соотношение между $\delta Q$ и $\delta P$;\\

$\delta Q=TdS=d\delta\deltaU+PdV=dH-VdP$;\\


$\delta Q=\left( \frac{\partial U}{\partial T} \right)_V dT+\left( \frac{\partial U}{\partial V} \right)_T+PdV=C_VdT+\left[\left( \frac{\partial U}{\partial V} \right)_T + P\right]dV=\left[\left( \frac{\partial U}{\partial V} \right)_T + P\right]dV=\left[\left( \frac{\partial U}{\partial V} \right)_T + P\right]\left( \frac{\partial V}{\partial P} \right)_T dP$\\

т.к. T=const и $dV=\left( \frac{\partial V}{\partial T} \right)_P dT+\left( \frac{\partial V}{\partial P} \right)_T dP=\left( \frac{\partial V}{\partial P} \right)_T dP$;\\

$dU=TdS-PdV, \left( \frac{\partial U}{\partial V} \right)_T =T\left( \frac{\partial S}{\partial V} \right)_T-P\Rightarrow$\\

$\delta Q=T\left( \frac{\partial S}{\partial V} \right)_T \left( \frac{\partial V}{\partial P} \right)_T dP=-T\left( \frac{\partial V}{\partial T} \right)_P dP=-TV_0\alpha dP$\\

$\triangle Q=-TV_0\alpha \triangle P\Rightarrow$\\

$\alpha=\frac{\triangle Q}{TV_0 \triangle P}=4.7\cdot10^{-4}$ К$^{-1}$\\

3) $\delta Q=0$ - адиабата S=const;\\

$A=\int\limits_1^2 PdV=\int\limits_1^2 \left( \frac{\partial V}{\partial P} \right)_S dP$\\

$\left( \frac{\partial V}{\partial P} \right)_S=\left( \frac{\partial V}{\partial P} \right)_T \frac{\left( \frac{\partial S}{\partial T} \right)_V}{\left( \frac{\partial S}{\partial T} \right)_P}=\left( \frac{\partial V}{\partial P} \right)_T\frac{C_V}{C_P}=\frac{1}{\gamma}\left( \frac{\partial V}{\partial P} \right)_T$\\

$\left( \frac{\partial V}{\partial P} \right)_T=V\beta_T$

$A=\frac{V\beta_T}{\gamma}\int\limits_1^2 PdP=\frac{V\beta_T}{2\gamma}(P_2^2-P_1^2)$;\\

$\beta_T=\frac{2\gamma A}{V(P_2^2-P_1^2)}=2.2\cdot10^{-10}$ Па$^{-1}$,\\

$\kappa=-\frac{\alpha}{\beta_T}=2\cdot10^{-6}$ Па/К




\newpage

{\underline\bf Задача 12.8 }

\begin{figure}[h]
%\begin{center}
\includegraphics[width=12cm]{12_8_0}
\end{figure}

\vspace{0.5cm}

$Адиабата \delta Q=0, \delta Q=dU+\delta A=C_x dT+q dX=0$, где $C_x$ - теплоемкость и q - поверхностное натяжение.

$\triangle T=-\frac{q\triangle X}{C_x}$;

$C_x=Cm=C\rho V=C\rho h\triangle X=C\rho h\cdot X_0$, где С-теплоемкость воды и $\triangle X=2X_0-X_0=X_0$\\

$\triangle T=-2q\frac{X_0}{C\rho h\cdot X_0}=-\frac{2q}{C\rho h}$

Из условия задачи q неизвестно, но известна производная $\frac{d\sigma}{dT}=-0.15$ дин/(см К).

В задаче $12.5*$ рассмотрен цикл Карно, в котором рабочем телом является мыльная пленка. В результате найдено соотношение  для q и $\frac{d\sigma}{dT}$ :\\

$\frac{d\sigma}{dT}=-\frac{q}{T}$, откуда $q=-T\frac{d\sigma}{dT}$;\\ Подставляя, получим:

$\triangle T=-\frac{2q}{C\rho h}=-2\frac{d\sigma}{dT}\frac{T}{C\rho h}=-0.02$ К при теплоемкости воды С=4174 Дж/(кг К).


\newpage


{\underline\bf Задача 5.42}

\begin{figure}[h]
%\begin{center}
\includegraphics[width=12cm]{5_42_0}
\end{figure}

\vspace{0.5cm}

$\delta Q=TdS=dU+PdV, C_V=\frac{\delta Q}{dT}=\left(\frac{dU}{dT}\right)_V=T\left(\frac{dS}{dT}\right)_V$;\\

$S=-\left(\frac{d\Psi}{dT}\right)_V=4AVT^3$\\

$C_V=T\left(\frac{dS}{dT}\right)_V=12AVT^3$, T-?\\

$P=-\left(\frac{d\Psi}{dV}\right)_T=AT^4, T=\left(\frac{P}{A}\right)^{\frac{1}{4}}$;\\

$C_V=12AVT^3=12AV\left(\frac{P}{A}\right)^{\frac{3}{4}}$\\

$C_V=12A^{\frac{1}{4}}P^{\frac{3}{4}}V=8.4\cdot10^4$ эрг/К;\\

$C_V^{\gamma}=8\cdot C_V^{id}$








\end{document} 